\documentclass[a4paper]{scrartcl}
\usepackage[utf8]{inputenc}
\usepackage{csquotes}
\usepackage{color}
\usepackage[american]{babel} % Hyph.: german, ngerman, spanish or deactivate for english
\usepackage[parfill]{parskip} % Activate to begin paragraphs with an empty line rather than an indent
\usepackage[sorting=none]{biblatex}
\usepackage{hyperref}
\bibliography{se_beitraege}
\definecolor{gray}{rgb}{0.5,0.5,0.5}
\newcommand*{\annotecite}[1]{\fullcite{#1}\\ {\color{gray}\citefield{#1}{annotation}}}

\hypersetup{colorlinks=true}

\title{FLOSSing proprietary code}
\subtitle{Seminarbeitrag, Beiträge zum Software Engineering}
\author{David Kaltschmidt (3782360)}

\begin{document}
\maketitle

I propose to present two papers on private-collective innovation\footnote{\citeauthor{stuermer2009extending} are using the term \emph{innovation} in the tradition of \textcite[Ch 3.B, p. 83]{schumpeter1939business} as being a new combination of production factors (invention) and its immediate adoption.} 
where private companies enlist the help of the FLOSS community to build a commercial product. 
Shared development costs via community contributions as well as boosted organizational learning are obvious benefits. 
But what are the drawbacks? 
The second paper focuses on the developer as a link between the company that employs him and the community.

%\annotecite{dabbish2012social}

\section*{\citetitle{stuermer2009extending}}
\fullcite{stuermer2009extending}

\begin{itemize}
\item private investment model of innovation vs. investment of resources to create public goods 
\item characterized by non-rivalry and non-exclusivity in consumption, intellectual property rights are forfeited and the resulting innovation is offered to the public for free
\item case study of Nokia Internet Tablet shows benefits and costs
\item study conducted semi-structured interviews, allowing participants the opportunity to narrate stories
\end{itemize}

\subsection*{Presentation}

\begin{itemize}
\item introduce private-collective innovation model
\item previously known benefits when following the model
\item bring attention to hidden costs
\item describe study/interviews, why was the product a suitable subject
\item present findings: benefits, costs, strategies to mitigate costs
\item provide 2-3 anecdotes from the interviews
\end{itemize}


\section*{\citetitle{henkel2009champions}}
\fullcite{henkel2009champions}

\begin{itemize}
\item the link between firms engaging in open source software (OSS) development and the OSS community is established by individual developers 
\item double allegiance to firm and OSS community
\item developers expose the firm to the risk of losing intellectual property
\item study done using interviews and large-scale survey
\item focus on potential principal-agent problems between developer and employer
\item biggest problem for companies: the Free Software ideologist
\end{itemize}

\subsection*{Presentation}

\begin{itemize}
\item introduce topic, actors, principles (e.g. ``Free Software ideology'')
\item programmer as bridge between two worlds
\item bring attention to principal-agent problems (information asymmetry and potentially diverging interests)
\item describe study/interviews, why was the product a suitable subject
\item present findings: revealing code may be in firm's best interest, management overestimates the risk and underestimates the benefits of openness
\item provide 2-3 anecdotes from the interviews
\end{itemize}


\printbibliography[keyword=innovation]

\end{document}  
